\documentclass{article}
\usepackage{graphicx}
\usepackage{amsmath}

\begin{document}

\newcommand{\arrvec}{\texttt{ArrVec} class}
\newcommand{\view_arrvec}{\texttt{View\_ArrVec} class}

\section{Skewblas Array and Vector Library Documentation}

The Skewblas library provides efficient array and vector operations, leveraging MKL (Math Kernel Library) for optimized performance. The documentation below details the implementation of key classes and functions.

\subsection{Class Overview}

\subsubsection{View\_ArrVec Class}
A view class that provides direct access to an array's elements without copying data. It stores the array pointer and its size, enabling efficient element-wise operations.

\subsubsection{\texttt{ArrVec} Class}
A wrapper class for arrays, supporting dynamic allocation and copy operations. It encapsulates memory management and provides methods for array manipulation.

\section{Function Descriptions}

\subsection{View\_ArrVec}

\begin{itemize}
    \item Initialization: Constructs a view from an existing array or initializes it with given dimensions.
    \item Assignment: Copies data from another view or array using \texttt{memcpy}.
    \item Swapping: Exchanges the view's array with another instance.
    \item Destructor: Releases memory when the view goes out of scope (should not be used manually).
\end{itemize}

\subsection{\texttt{ArrVec}}

\begin{itemize}
    \item Initialization: Allocates memory for an array of specified size.
    \item Copying: Copies data from another array or view, managing memory allocation as needed.
    \item Assignment: Copies data into the current array using \texttt{memcpy}.
    \item Element Access: Provides access to elements via index using overloaded operators.
\end{itemize}

\subsection{Element-wise Operations}

\begin{itemize}
    \item Ones(): Sets all elements to 1.
    \item Zeros(): Initializes the array with zeros.
    \item Random Values: Generates random values within a specified range.
\end{itemize}

\subsection{I/O Functions}

\begin{itemize}
    \item Printing functions: Output the array contents in formatted text or to a file, optionally truncated.
    \item File operations: Write array data to a specified file with customizable formatting.
\end{itemize}

\section{Example Usage}

Using the provided classes:

\begin{code}
// Create an ArrVec instance
ArrVec<double> vec(5);
vec.Assign(&someArray);

// Use view for direct access
View_ArrVec<double> view = vec;
double element = view(2); // Accesses the third element

// Print the array
vec.printf("Results:", 10, "%1.3f\t");
\end{code}


\section{ColMat Class Documentation}

The ColMat class is a wrapper for an array of elements of type ELEM_TYPE. It provides methods for array manipulation, including copying data, swapping arrays, assigning values, and element-wise operations. The class also includes I/O functions for printing the array contents.

\subsection{Constructor}
\begin{itemize}
    \item Input: Two integers, row (number of rows) and column (number of columns)
    \item Output: A new ColMat instance with an array of size row*column
    \item Description: Initializes the array with the specified dimensions. If no arguments are provided, defaults to a zero-sized array.
\end{itemize}

\subsection{Destructor}
\begin{itemize}
    \item Input: None
    \item Output: Deletes the allocated memory for the array
    \item Description: Releases memory when the object goes out of scope. Should not be called manually.
\end{itemize}

\subsection{Copy Constructor}
\begin{itemize}
    \item Input: Another ColMat instance
    \item Output: A copy of the input ColMat instance
    \item Description: Copies data from another ColMat object into the current instance. The constructor calls the copy method to perform the operation.
\end{itemize}

\subsection{Copy Method}
\begin{itemize}
    \item Input: Another ColMat instance
    \item Output: Copies data from the input ColMat instance into the current array
    \item Description: Uses memcpy to transfer the entire array contents. This method is called by the copy constructor and other assignment operations.
\end{itemize}

\subsection{Swap Method}
\begin{itemize}
    \item Input: Another ColMat instance
    \item Output: Swaps the data pointers of both ColMat instances
    \item Description: Exchanges the array pointers between two instances. Also swaps row and column counts.
\end{itemize}

\subsection{Assignment Operator (operator=)}
\begin{itemize}
    \item Input: Another ColMat instance
    \item Output: Returns a reference to the current ColMat instance after copying data
    \item Description: Creates a temporary instance, swaps it with the current one, and returns the original instance. This approach avoids deep copying unless necessary.
\end{itemize}

\subsection{Assign Method}
\begin{itemize}
    \item Overload 1: Takes another ColMat instance, copies its data using the copy method
    \item Overload 2: Takes a VIEW_TYPE (another view class) and uses ld (leading dimension), then r and c to assign data
    \item Output: Assigns data from the source array into the current array
    \item Description: Handles assignment from different types of sources, ensuring efficient memory management.
\end{itemize}

\subsection{Copyto Method}
\begin{itemize}
    \item Overload 1: Copies data from another ColMat instance into a new array
    \item Overload 2: Uses memcpy to transfer data directly into the target array
    \item Output: Copies data into the target array
    \item Description: Similar to Assign but focuses on copying data without modifying the source array.
\end{itemize}

\subsection{Element-wise Operations}
\begin{itemize}
    \item Ones(): Sets all elements to 1
    \item Zero(): Initializes the array with zeros
    \item Idmt(): Initializes the array as an identity matrix (1s on the diagonal)
    \item Rand(): Generates random values within a specified range using a random engine
\end{itemize}

\subsection{I/O Functions}
\begin{itemize}
    \item fprintf and printf: Print the array contents in formatted text or to a file
    \item These functions handle edge cases like truncating output based on specified limits
\end{itemize}


\end{document}